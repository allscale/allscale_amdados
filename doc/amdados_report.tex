\documentclass[]{article}
\usepackage{amsmath}
\usepackage{amssymb}
\usepackage{xcolor}

%opening
\title{Allscale AMDADOS Application \\ Techinal Report}
\author{Albert Akhriev}

\begin{document}
\maketitle
%\begin{abstract}
%\end{abstract}

%%%%%%%%%%%%%%%%%%%%%%%%%%%%%%%%%%%%%%%%%%%%%%%%%%%%%%%%%%%%%%%%
\section{Overview}
%%%%%%%%%%%%%%%%%%%%%%%%%%%%%%%%%%%%%%%%%%%%%%%%%%%%%%%%%%%%%%%%



%%%%%%%%%%%%%%%%%%%%%%%%%%%%%%%%%%%%%%%%%%%%%%%%%%%%%%%%%%%%%%%%
\section{Working with Amdados application}
%%%%%%%%%%%%%%%%%%%%%%%%%%%%%%%%%%%%%%%%%%%%%%%%%%%%%%%%%%%%%%%%

\subsection{Building the application.}
\begin{enumerate}
\item The application executable must be available before any test (even in Python) is run.
\item We recommend the standard way for building the application presented in the script ``\textit{standard.build.sh}'' in the project root folder.
\item Another useful script ``./scripts/download.sh'' downloads the latest Allscale API and the Armadillo library for unit tests.
\item We do \textit{not} recommend the development script for building the application ``\textit{mybuild}'', which relies on ramdisk and other development specific features. Just for completeness, we provide script's options: \texttt{-f} clears any previous build and starts from scratch; \texttt{-r/-d} release/debug mode; \texttt{-t} runs tests after building the project. For example: \texttt{./mybuild -f -r -t}.
\end{enumerate}

\subsection{Application parameters.}
\begin{enumerate}
\item The application is controlled by the configuration file. The default one can be found in the project root folder under the name ``\textit{amdados.conf}''. The most interesting parameters are: the integration period and the number of sub-domains in either dimension. 
\item Not everything can be controlled by configuration file. For example, the size of sub-domain is hard-coded because of using of templates in C++ grid implementation.
\item The flow model is also hard-coded (see the function \texttt{Flow()} in the file ``\textit{scenario\_simulation.cpp}'' and corresponding function in the Python code). The reason for this is to get away of any specific format of flow data representation while focusing on Allscale API piece.
\item The same is true for sensor locations. Currently, user has to specify a fraction of nodal points occupied by sensors and then the location are (pseudo) randomly generated.
\item The default configuration file ``\textit{amdados.conf}'' contains brief description of each parameter sufficient to match it to the source code.
\end{enumerate}

\subsection{The application structure}
\begin{enumerate}
\item In order to demonstrate a data assimilation process we need a ``true'' solution or, in other words, a ``true state of nature''. The reason is twofold. First, this ground-truth is used for the accuracy assessment of data assimilation. Second, the \textit{input data} ``measured'' at sensor locations across the domain govern the simulation by pushing it towards the ``true'' solution.
\item In Amdados use-case scenario, the \textit{input data are simulated} by direct forward solver (see equation (\ref{?})) in the entire domain. The solver is written in Python. For simplicity, the Python forward solver does not implement any sub-domain partitioning, operating in the entire domain directly, but uses the same number of nodal points as the C++ Amdados application, the same integration period and the same flow model, see the functions \texttt{Flow()} in both C++ and Python implementations.
\item The Python code, implemented in ``\textit{python/ObservationsGenerator.py}'', reads the configuration file (assuming the C++ application will use the same file later on) to set-up geometry and initial conditions. Then it runs the forward solver storing two large files.
\item Those files are: (1) values of the state field at all sensor locations recorded every time step; this file has prefix ``\textit{analytic}'' in its name for historical reasons (it might contain data of analytic as well as simulated solution); (2) the file of a number of entire fields with prefix ``\textit{true\_field}''\footnote{Parameter \textit{write\_num\_fields}.}. The latter file contains full (ground-truth) state fields for presentation and comparison against the data assimilation solution. It is important to emphasize that these files can grow up to dozens of gigabytes in case of a large problem. For this reason, we record solution values only at sensor locations on each time step and a quite small number of full fields over the course of simulation to save memory.
\item Input data for C++ \texttt{Amdados} application:
    \begin{itemize}
    \item Text file of sensor locations: ``\textit{sensors*.txt}'';
    \item Text file of \textit{observations} at sensor locations: ``\textit{analytic*.txt}''.
    \end{itemize}
    The first of these file is generated by \texttt{Amdados} application running in
    special scenario\footnote{Sensors' generation by Python is too slow, so we use C++ implementation instead.}, e.g.:
    $$
    \mbox{\small\texttt{build/app/amdados --scenario sensors --config config-file-name}}
    $$
    For convenience, is a file of sensor locations was not found in the output directory\footnote{Parameter \textit{output\_dir}.} upon launching the script ``\textit{python/ObservationsGenerator.py}'', the \texttt{Amdados} application will be automatically run in aforementioned scenario to generate sensors (pseudo) randomly seeded inside the domain.
\item In order to distinguish simulations with different settings the naming convention was adopted, which is best explained by example:
$$
\mbox{\small\texttt{analytic\_Nx176\_Ny176\_Nt1225.txt}},
$$
where \textit{analytic} is the \textsf{data type prefix}, $N_x=176$ is the total number of nodal points in $x$ dimension, $N_y=176$ is the total number of nodal points in $y$ dimension, $N_t=1225$ is number of time integration step, and \textit{txt/bin} is the extension of a text or binary file respectively. The number of time steps is omitted in the name of sensor locations file because sensors do not change their positions over time.
\end{enumerate}

\subsection{Running the Amdados simulation}
\subsubsection{Single run}
\begin{enumerate}
\item Step into the project root directory.
\item Create a configuration file using whatever settings you need. It is not recommended to modify the default file ``\textit{amdados.conf}'' or tweak any parameter except the number of sub-domains in either direction and the integration time (unless you understand all the implications). Also, note that the size of a sub-domain must coincide with the hard-coded size in C++ code, where it is used for template instantiation. We advice to use ``\textit{amdados.conf}'' as a starting point.
\item Build \texttt{Amdados} executable in ``Release'' mode:
$$
\mbox{\small\texttt{bash standard.build.sh}}
$$
The building script extracts the output directory from ``\textit{amdados.conf}''. If your choice is different, please, modify the simple file ``\textit{standard.build.sh}'' accordingly.
\item If you want to get rid of any previous stuff in the output folder it is worth to do this right after building the executable. Note, all the simulation artefacts will go into the output directory. 
\item Run the Python script to generate a file of sensor locations (if it is missed), the file of observations and the file of full-field snapshots:
$$
\mbox{\small\texttt{python3 python/ObservationsGenerator.py --config your.conf}}
$$
If not specified, the configuration file ``\textit{amdados.conf}'' will be used. The command \texttt{--help} lists all the options. Upon completion the following files will be created in the output folder:
$$\mbox{\small\texttt{sensors\_Nx*\_Ny*.txt}}$$
$$\mbox{\small\texttt{analytic\_Nx*\_Ny*\_Nt*.txt}}$$
$$\mbox{\small\texttt{true\_field\_Nx*\_Ny*\_Nt*.bin}}$$
where the values under symbols '$*$' depend on current settings.
\end{enumerate}
\subsubsection{Size scalability test}
\begin{enumerate}
\item fff
\end{enumerate}
\subsubsection{Multi-threading scalability test}
\begin{enumerate}
\item fff
\end{enumerate}

\end{document}
