\documentclass[]{article}
\usepackage{amsmath}
\usepackage{amssymb}
\usepackage{xcolor}

%opening
\title{Allscale AMDADOS Application \\ Techinal Report}
\author{Albert Akhriev}

\begin{document}
\maketitle
%\begin{abstract}
%\end{abstract}

\section{Overview}

\section{Working with Amdados application}

\subsection{Building the application.}
\begin{enumerate}
\item The application executable must be available before any test (even in Python) is run.
\item We recommend the standard way for building the application presented in the script ``standard.build.sh'' in the project root folder.
\item Another useful script is ``./scripts/download.sh'' which downloads the latest Allscale API and the Armadillo library for unit tests.
\item We do \textit{not} recommend the development script for building the application ``mybuild'', which relies on ramdisk and other development specific features. Just for completeness, we provide script's options: \texttt{-f} clears any previous build and starts from scratch; \texttt{-r/-d} release/debug mode; \texttt{-t} runs tests after building the project. For example: \texttt{./mybuild -f -r -t}.
\end{enumerate}

\subsection{Application parameters.}
\begin{enumerate}
\item The application is controlled by the configuration file. The default one can be located in the project root folder under the name ``amdados.conf''. The most interesting parameters are: the integration period and the number of sub-domains in either dimension. 
\item Not everything can be controlled by configuration file. For example, the size of sub-domain is hard-coded because of using of templates in grid implementation.
\item The flow model is also hard-coded (see the function \texttt{Flow()} in the file ``scenario\_simulation.cpp'' and corresponding function in the Python code). The reason for this is to get away of any specific format of flow data representation while focusing on Allscale API piece.
\item The same is true for sensor locations. Currently, user has to specify a fraction of nodal points occupied by sensors and then the location are (pseudo) randomly generated.
\item The default configuration file ``amdados.conf'' contains brief description of each parameter sufficient to match it to the source code.
\end{enumerate}

\subsection{Running the Amdados simulation}


\end{document}
